%%% Template originaly created by Karol Kozioł (mail@karol-koziol.net) and modified for ShareLaTeX use




\documentclass[a4paper,11pt]{article}

\usepackage[T1]{fontenc}
\usepackage[utf8]{inputenc}
\usepackage{graphicx}
\usepackage{xcolor}
\usepackage{ngerman}
\usepackage[tx]{sfmath}
\usepackage{calc}
\usepackage{lastpage}
%\usepackage{ifthen}
\usepackage{xifthen}
\usepackage{multicol}
\renewcommand\familydefault{\sfdefault}
\usepackage{tgheros}
\usepackage{intcalc}

\usepackage{amsmath,amssymb,amsthm,textcomp}
\usepackage{enumerate}
\usepackage{multicol}
\usepackage{tikz}

\usepackage{geometry}
\geometry{total={210mm,297mm},
left=25mm,right=25mm,%
bindingoffset=0mm, top=20mm,bottom=20mm}


%\newcommand{\projektid}{\_\_}
%\newcommand{\iban}{DE08 1001 0010 0745 3161 07}
%\newcommand{\empfaenger}{Michael Braun (jr)}
%\newcommand{\hv}{Lukas Staab am 2017-10-20}
%\newcommand{\kv}{Lukas Staab am 2017-10-20}
%\newcommand{\projektname}{TestTestTest999}
%\newcommand{\recht}{Referat Ehrenamt/2017-01-16 (StuRa: 1234)}
%\newcommand{\betrag}{660}
%\newcommand{\posten}{B1,B1,B1,B2}
%\newcommand{\einnahmen}{0,0,0,0}
%\newcommand{\ausgaben}{10,500,50,100}
%\newcommand{\hhptitel}{0 1001 2,0 2115 2,0 2231 2,0 2231 2}
%\newcommand{\picpaths}{1/59e93cfbbea7c.pdf}

\linespread{1.3}

\newcommand{\linia}{\rule{\linewidth}{0.5pt}}

% custom theorems if needed
\newtheoremstyle{mytheor}
    {1ex}{1ex}{\normalfont}{0pt}{\scshape}{.}{1ex}
    {{\thmname{#1 }}{\thmnumber{#2}}{\thmnote{ (#3)}}}

\theoremstyle{mytheor}
\newtheorem{defi}{Definition}

% my own titles
\makeatletter
\renewcommand{\maketitle}{
\begin{center}
\vspace{2ex}
{\huge \textsc{\@title}}
\vspace{1ex}
\\
\linia\\
\@author \hfill \@date
\vspace{4ex}
\end{center}
}
\makeatother
%%%

% custom footers and headers
\usepackage{fancyhdr}
\pagestyle{fancy}
\lhead{}
\chead{}
\rhead{}
\lfoot{\footerstring}
\cfoot{}
\rfoot{Seite \thepage{} von \pageref{LastPage}}
\renewcommand{\headrulewidth}{0pt}
\renewcommand{\footrulewidth}{0pt}

%
% all section titles centered and bolded
\usepackage{sectsty}
\allsectionsfont{\centering\bfseries\large}
%
% add section label
\renewcommand\thesection{}
%

%%%----------%%%----------%%%----------%%%----------%%%

\begin{document}

\title{Zahlungsanweisung}

\author{StuRa der TU Ilmenau}

\date{nach ThürStudFVO §12 (2)}

\maketitle

\section{Allgemeine Angaben}

\begin{enumerate}[I]
\item \textbf{Zahlungsempfänger}\hfill \empfaenger
\item \textbf{IBAN} \hfill\iban
\item \textbf{Gesamtbetrag} \hfill\betrag{} EUR
\item \textbf{Rechtsgrundlage} \hfill \ifthenelse{\isempty\recht}{\parbox[b]{0.33\linewidth}{\strut{\hrule}}}{\recht}
\item \textbf{Zugehöriges Projekt} \hfill [\projektid] \projektname
\item \textbf{Datum der Auszahlung} \hfill \ifthenelse{\isempty{\datumauszahlung{}}}{\datumauszahlung}{\parbox[b]{0.05\linewidth}{\strut{\hrule}}.\parbox[b]{0.05\linewidth}{\strut{\hrule}}.\the\year}
\end{enumerate}
\vspace{1cm}
\parbox[b]{0.4\linewidth}{% size of the first signature box
    \strut 
    \textbf{Sachliche Richtigkeit} \\[1.25cm]% This 2cm is the space for the signature under the names
    \hrule
    \vspace{0.25cm}
    (\ifthenelse{\isempty \hv}{Haushaltsverantwortliche/r}{\hv})} 
\hspace{1cm} % distance between the two signature blocks 
\parbox[b]{0.4\linewidth}{% ...and the second one
    \strut 
    \textbf{Rechnerische Richtigkeit} \\[1.25cm]% This 2cm is the space for the signature under the names
    \hrule 
    \vspace{0.25cm}
    (\ifthenelse{\isempty \kv}{Kassenverantwortliche/r}{\kv})}
    \par\vspace{1cm} 

\tikzset{
  font={\fontsize{29pt}{12}\selectfont}}
%Aufschlüsselung der Belege
\section{Aufschlüsselung}
\begin{multicols}{4}
	\begin{flushright}	
	\noindent \textbf{Beleg} \\
	\foreach \zeile in \posten {
		 \zeile \\
	}	
	\end{flushright}
	\begin{flushright}
	\noindent\textbf{HHP-Titel} \\
	\foreach \zeile in \hhptitel {
		 \zeile  \\
	}
	\end{flushright}
	\begin{flushright}
	\noindent \textbf{Einnahmen} \\ 
	\foreach \zeile in \einnahmen {
		\zeile{} EUR\\
	}
	\end{flushright}
	\begin{flushright}
	\noindent \textbf{Ausgaben} \\
	\foreach \zeile in \ausgaben {
		 \zeile{} EUR\\
	}
	\end{flushright}
	
\end{multicols}
\newpage
%Belege
\tikzset{
  font={\fontsize{29pt}{12}\selectfont}}
\foreach \nr/\picname in \picpaths
{
	%% bei mehrseitigen Belegen erst alle originale und dann alle Kopien 
	%% bei einseitigen Belegen, beides auf einer Seite
	
	\pdfximage{\picname}
    %Switch case der Anzahl Seiten des Beleges
    
                
    \begin{tikzpicture}[remember picture,overlay]
    	 %rechteck
         \draw [draw=black,line width=5mm,opacity=0.3] (16.5,-10) rectangle (-1,-1.25);
        	%belegnummer		
		\node [opacity=1,anchor=north west,xshift=0.95\textwidth, yshift=1.25cm] {B\nr};
			%text im Rechteck
		\node [opacity=0.3,anchor=north west, yshift=-2.5cm] {Hier unten abgebildeten };
		\node [opacity=0.3,anchor=north west, yshift=-3.6cm] {Beleg \nr{} antackern (Original)};
		\node [opacity=0.3,anchor=north west, yshift=-5.5cm] {Falls dieser Beleg ein A4 Beleg };
		\node [opacity=0.3,anchor=north west, yshift=-6.6cm] {ist, hefte das Original vor};
		\node [opacity=0.3,anchor=north west, yshift=-7.7cm] {dieser Seite ab};
         \node [anchor =north east, inner sep=0pt,outer sep=0pt,yshift=-0.45\paperheight] at (current page.north east) {\fbox{\includegraphics[page=1,angle=90,width=0.925\paperwidth,height=0.45\paperheight]{\picname}}};
	        \node [rotate=90, opacity=0.75,anchor=north west,xshift=-0.73\paperheight,yshift=2.92cm]{ Beleg \nr - Seite 1};
    \end{tikzpicture}
    \newpage
    \ifthenelse{\the\pdflastximagepages > 1}{
        	\foreach \index in {2,...,\the\pdflastximagepages}
				{%iteriere über die Seiten des Beleg pdfs
				\ifthenelse{\not \isodd{\index}}{
					\begin{tikzpicture}[remember picture,overlay]  
   						\node [opacity=1,anchor=north west,xshift=0.95\textwidth, yshift=1.75cm] {B\nr}; 
      					\node [anchor =north east, inner sep=0pt,outer sep=0pt,yshift=-2cm] at (current page.north east) {\fbox{\includegraphics[page=\index,angle=90,width=0.925\paperwidth,height=0.415\paperheight]{\picname}}};
		    			\node [rotate=90, opacity=0.75,anchor=north west,xshift=-0.275\paperheight,yshift=2.92cm] { Seite \index};
  					\end{tikzpicture}
				}{ %else unterer Teil des Blattes
					\begin{tikzpicture}[remember picture,overlay]  
      					\node [anchor =north east, inner sep=0pt,outer sep=0pt,yshift=-14.55cm] at (current page.north east) {\fbox{\includegraphics[page=\index,angle=90,width=0.925\paperwidth,height=0.415\paperheight]{\picname}}};
		    			\node [rotate=90, opacity=0.75,anchor=north west,xshift=-0.65\paperheight,yshift=2.92cm] { Seite \index};
  					\end{tikzpicture}
					\newpage
				} %if end
   		  		
		}{} %else end if page > 1

    \newpage
	}
}


\end{document}
